\documentclass[12pt]{article}

% Essential packages for math and symbols
\usepackage{amsmath}
\usepackage{amssymb}
\usepackage{amsfonts}
\usepackage{mathtools}

% For better typography and language support
\usepackage[utf8]{inputenc}
\usepackage[T1]{fontenc}
\usepackage{microtype}

% For enumerated lists and itemize environments
\usepackage{enumitem}

% For custom math operators and symbols
\DeclareMathOperator*{\argmin}{arg\,min}
\DeclareMathOperator*{\argmax}{arg\,max}

% Custom math commands for frequently used notation
\newcommand{\R}{\mathbb{R}}
\newcommand{\N}{\mathbb{N}}
\newcommand{\ip}{\text{ip}}
\newcommand{\mail}{\text{mail}}
\newcommand{\total}{\text{total}}

% Margins 1 inch
\usepackage[letterpaper, margin=1in]{geometry}

% Document begins
\begin{document}

\subsubsection{Examining measurement error via artificial mail-in districts}
\label{sec:artificial_districts}

To evaluate the measurement error introduced by our mail-in vote distribution procedure, we construct \emph{artificial} mail-in districts and compare municipalities’ \emph{total} votes (in-person plus mail-in) under our distribution approach versus their \emph{observed} total votes. Let $\mathcal{I}$ be the set of municipalities that \emph{do not} share a mail-in district (i.e., each municipality $i \in \mathcal{I}$ has its own unique mail-in vote count). Denote by $w_i$ the number of polling card voters in municipality~$i$, let $Y_i^{\text{ip}}$ be its in-person votes, and $Y_i^{\text{mail}}$ its observed mail-in votes. Hence, the \emph{observed total} votes are 
\[
    Y_i^{\text{total}} \;=\; Y_i^{\text{ip}} \;+\; Y_i^{\text{mail}}.
\]
We proceed as follows:

\begin{enumerate}
    \item \textbf{Subset the data:} Restrict the dataset to all municipalities $i \in \mathcal{I}$. For each of these, we observe $Y_i^{\text{ip}}$, $Y_i^{\text{mail}}$, and $w_i$.

    \item \textbf{Create $D$ artificial mail-in districts:} Partition the municipalities in $\mathcal{I}$ into $D$ groups (districts). We consider two approaches:
    \begin{itemize}
        \item \textit{Random assignment:} Assign each municipality $i$ to one of the $D$ districts with equal probability.
        \item \textit{K-means clustering:} Use centroid coordinates of municipalities to create spatially contiguous districts via a $k$-means algorithm. This more closely reflects the geographic adjacency typical of real mail-in districts.
    \end{itemize}

    \item \textbf{Aggregate mail-in votes and redistribute:} For each artificially created district $d \in \{1,\dots,D\}$, let
    \[
        Y_d^{\text{mail}} \;=\; \sum_{i \in d} Y_i^{\text{mail}}
    \]
    be the \emph{total} artificial mail-in votes. We then redistribute $Y_d^{\text{mail}}$ back to each municipality $i \in d$ based on polling card voters:
    \[
        \widetilde{Y}_i^{\text{mail}} 
        \;=\; 
        \frac{w_i}{\sum_{j \in d} w_j} 
        \; Y_d^{\text{mail}},
    \]
    mirroring our procedure in Section \ref{sec:mailin}. Next, we form an \emph{artificial} total vote count for $i$,
    \[
        \widetilde{Y}_i^{\text{total}} 
        \;=\; 
        Y_i^{\text{ip}} 
        \;+\;
        \widetilde{Y}_i^{\text{mail}},
    \]
    which combines in-person votes and \emph{redistributed} mail-in votes.

    \item \textbf{Compare actual vs.\ artificial totals:} For each municipality $i$, compute the discrepancy
    \[
        \Delta_i 
        \;=\; 
        Y_i^{\text{total}}
        \;-\; 
        \widetilde{Y}_i^{\text{total}}
        \;=\;
        \Bigl(Y_i^{\text{ip}} + Y_i^{\text{mail}}\Bigr)
        \;-\;
        \Bigl(Y_i^{\text{ip}} + \widetilde{Y}_i^{\text{mail}}\Bigr)
        \;=\;
        Y_i^{\text{mail}} - \widetilde{Y}_i^{\text{mail}},
    \]
    which measures how much the \emph{artificial} total deviates from the \emph{observed} total. This offers a direct estimate of the measurement error induced by our polling-card-based allocation method under different districting scenarios.
\end{enumerate}

This exercise reveals how well the mail-in distribution method recovers actual total votes when municipalities’ mail-in votes are artificially aggregated and then reallocated. The main drawback of this method is that the districts we generate do not actually exist. 

We conduct the procedure outlined above for the 2021 election results using different numbers of artificial districts, more specifically $D \in \{30, 100, 300, 1000\}$. As $D$ increases, the number of municipalities per district decreases. For each district size, we calculate the median of $\Delta_i$, as well as the 25th and 75th percentiles (i.e., $Q_{25}(\Delta_i)$ and $Q_{75}(\Delta_i)$). We repeat this for the six main parties. The results are shown in figure \ref{fig:artificial_districts}. We find that measurement error are generally small for all parties other than the AfD.



\paragraph{When is measurement error larger, and what about the sign of \(\Delta_i\)?}

Recall from Section~\ref{sec:artificial_districts} that the measurement error for municipality~\(i\) in a mail-in district~\(d\) is given by
\[
    \Delta_i
    \;=\;
    Y_i^{\text{mail}}
    \;-\;
    \widetilde{Y}_i^{\text{mail}}
    \;=\;
    M_i 
    \;-\;
    \frac{W_i}{\sum_{j \in d} W_j} 
    \sum_{k \in d} M_k,
\]
where \(M_i = Y_i^{\text{mail}}\) is the \emph{actual} mail-in votes of municipality~\(i\), and \(W_i\) is the number of polling card voters in \(i\). Let
\[
    M_d = \sum_{k \in d} M_k
    \quad\text{and}\quad
    W_d = \sum_{k \in d} W_k.
\]
Then we can rewrite
\[
    \Delta_i
    \;=\;
    M_i \;-\; \frac{W_i}{W_d}\,M_d.
\]
Defining the \emph{mail-in vote ratio} 
\[
    a_i \;=\; \frac{M_i}{W_i}
    \quad\text{and}\quad
    a_d \;=\; \frac{M_d}{W_d},
\]
we obtain 
\[
    \Delta_i 
    \;=\; 
    W_i \,\bigl(a_i - a_d\bigr).
\]
Hence, two main factors jointly determine \(\Delta_i\): the gap \(\bigl(a_i - a_d\bigr)\) and the size of \(W_i\).

\subparagraph{Magnitude of the measurement error.}
Measurement error tends to grow larger when:
\begin{enumerate}
  \item \textit{High variance in mail-in vote ratios.} 
  If \(a_i\) differs considerably from the district average \(a_d\), the term \(\bigl(a_i - a_d\bigr)\) becomes large in absolute value. Thus, municipalities whose mail-in votes per polling card deviate sharply from the “typical” district-level pattern incur bigger \(\lvert \Delta_i\rvert\).

  \item \textit{Large disparity in \(\boldsymbol{W_i}\).}
  Even moderate gaps \(\lvert a_i - a_d\rvert\) can yield large errors if \(W_i\) is large. Intuitively, a big municipality with a polling-card-based ratio far from the district average experiences a more pronounced discrepancy in allocated votes.

  \item \textit{Heterogeneous municipalities forced into one district.}
  If a district contains municipalities with very different voting behaviors (different \(a_i\) values), the “average-based” allocation will fit poorly, amplifying measurement error for outliers.
\end{enumerate}

\subparagraph{Sign of \(\Delta_i\).}
From
\[
    \Delta_i \;=\; W_i(a_i - a_d),
\]
the sign of \(\Delta_i\) directly depends on whether \(a_i\) (municipality~\(i\)’s mail-in ratio) is above or below \(a_d\) (the district average):
\begin{itemize}
    \item If \(a_i > a_d\), then \(\Delta_i > 0\). 
    This means the actual mail-in votes \(M_i\) exceed the amount allocated under the polling-card procedure, so the procedure \emph{under-allocates} votes to \(i\).
    \item If \(a_i < a_d\), then \(\Delta_i < 0\). 
    Here, the municipality’s actual mail-in votes are \emph{lower} than what our allocation would assign, so the procedure \emph{over-allocates} votes to \(i\).
\end{itemize}

\subparagraph{Is there a district- or global-level systematic over/undercount?}
Interestingly, if we sum \(\Delta_i\) over all municipalities in a (real or artificial) district~\(d\),
\[
    \sum_{i \in d} \Delta_i
    \;=\;
    \sum_{i \in d} 
    \Bigl(
      M_i
      \;-\;
      \frac{W_i}{W_d} M_d
    \Bigr)
    \;=\;
    \sum_{i \in d} M_i
    \;-\;
    M_d \sum_{i \in d} \frac{W_i}{W_d}
    \;=\;
    M_d - M_d \cdot 1
    \;=\;
    0.
\]
Hence, across the entire district, the positive and negative \(\Delta_i\) values sum to zero. This means that although \emph{some} municipalities within a district might be systematically over-allocated, others will be under-allocated, balancing out to no net district-level bias. Of course, which particular municipalities are systematically over- or undercounted will depend on the sign of \(\bigl(a_i - a_d\bigr)\) and on \(W_i\). 

Overall, the procedure itself does not globally inflate or deflate mail-in votes, but it can cause substantial local misallocations if (i) municipalities differ markedly in their ratio of mail-in votes to polling card voters, or (ii) the size of municipalities (in terms of \(W_i\)) interacts strongly with those differences.

\paragraph{Vote shares and their measurement error.}
So far, we have focused on the \emph{levels} of votes (e.g., mail-in votes or total votes) and the resulting discrepancies from our allocation procedure. In many empirical applications, however, the quantity of interest is the \emph{vote share} for a given party, defined by:
\[
    s_i(p)
    \;=\;
    \frac{X_i(p)}{\sum_{r} X_i(r)},
\]
where \(X_i(p)\) is the number of votes for party \(p\) (mail-in plus in-person) in municipality~\(i\), and the denominator is the total votes across all parties \(r\). When we use the allocation method, \(X_i(p)\) is replaced by its \emph{allocated} version \(\widetilde{X}_i(p)\). Thus, the observed share is
\[
    s_i^{\text{obs}}(p)
    \;=\;
    \frac{X_i^{\text{obs}}(p)}{\sum_{r} X_i^{\text{obs}}(r)},
    \quad
    \text{where}
    \quad
    X_i^{\text{obs}}(p)
    \;=\;
    I_i(p) + M_i(p),
\]
and \(I_i(p)\) and \(M_i(p)\) denote in-person and mail-in votes for party~\(p\) in municipality~\(i\). Meanwhile, under the allocation approach,
\[
    \widetilde{s}_i(p)
    \;=\;
    \frac{X_i^{\text{agg}}(p)}{\sum_{r} X_i^{\text{agg}}(r)},
    \quad
    \text{where}
    \quad
    X_i^{\text{agg}}(p)
    \;=\;
    I_i(p) + \widetilde{M}_i(p).
\]
Here, \(\widetilde{M}_i(p)\) is the (re-)allocated mail-in votes for party~\(p\) in municipality~\(i\), constructed via the polling-card-based weighting (see Section~XX).

\subparagraph{Definition of the vote-share error.}
We thus define the vote-share error for party~\(p\) in municipality~\(i\) as
\[
    \Delta_i(p)
    \;=\;
    s_i^{\text{obs}}(p)
    \;-\;
    \widetilde{s}_i(p)
    \;=\;
    \frac{X_i^{\text{obs}}(p)}{\sum_{r} X_i^{\text{obs}}(r)}
    \;-\;
    \frac{X_i^{\text{agg}}(p)}{\sum_{r} X_i^{\text{agg}}(r)}.
\]
Unlike the case of \emph{vote levels}, the denominator here also changes under the allocation procedure, because
\[
    \sum_{r} X_i^{\text{agg}}(r)
    \;=\;
    \sum_{r}
    \bigl(I_i(r) + \widetilde{M}_i(r)\bigr)
    \;\neq\;
    \sum_{r}
    \bigl(I_i(r) + M_i(r)\bigr)
    \;=\;
    \sum_{r} X_i^{\text{obs}}(r).
\]
Hence the measurement error in vote shares depends on two forces:
\begin{enumerate}
    \item How the allocation procedure changes the \emph{numerator} (i.e., how much of party~\(p\)’s mail-in votes is gained or lost by municipality~\(i\)).
    \item How the procedure changes the \emph{denominator} (i.e., the total votes across \emph{all} parties in municipality~\(i\)).
\end{enumerate}

\subparagraph{Scenarios producing consistently higher or lower allocated vote shares.}
To see why the sign of \(\Delta_i(p)\) could be systematically positive or negative, consider these intuitive scenarios:

\begin{itemize}
  \item \textit{Party \(p\) is ``overrepresented'' in mail-in votes at municipality~\(i\).} 
  Suppose \(i\) is a stronghold for party~\(p\), with a disproportionately large fraction of mail-in votes going to \(p\). Under the \emph{district-average} allocation, \(\widetilde{M}_i(p)\) is determined by the polling-card weight of \(i\) and by the \emph{district-wide} distribution of mail-in votes for \(p\). If \(i\) is \textit{above} the district average for \(p\), then the procedure \emph{under-allocates} \(p\)’s mail-in votes to \(i\). This makes \(X_i^{\text{agg}}(p) < X_i^{\text{obs}}(p)\), driving the numerator of \(\widetilde{s}_i(p)\) downward. Meanwhile, other parties might be \emph{over}-allocated if \(i\) was also relatively weak for them in the actual data. Consequently, the denominator could rise or fall, but typically \(\Delta_i(p)\) will be \emph{negative} if the strong party is under-allocated.

  \item \textit{Party \(p\) is ``underrepresented'' in mail-in votes at municipality~\(i\).}
  Conversely, if \(i\) is \emph{below} the district average for party~\(p\), the procedure might \emph{over-allocate} mail-in votes for \(p\). That is, \(\widetilde{M}_i(p) > M_i(p)\). This can push \(X_i^{\text{agg}}(p)\) above the actual \(X_i^{\text{obs}}(p)\), yielding \(\Delta_i(p) > 0\). In effect, the district-based weighting artificially boosts \(p\) in \(i\).

  \item \textit{Changes in the denominator for all parties.} 
  Because vote shares depend on the total votes across parties in \(i\), the reallocation of mail-in votes for \emph{other} parties also influences \(\Delta_i(p)\). Even if \(\widetilde{M}_i(p)\) and \(M_i(p)\) are close, if \(\widetilde{M}_i(q)\) for other parties \(q \ne p\) is inflated, then \(\sum_{r}X_i^{\text{agg}}(r)\) can become larger than \(\sum_{r}X_i^{\text{obs}}(r)\). This tends to \emph{reduce} \(\widetilde{s}_i(p)\), pushing \(\Delta_i(p)\) more negative.
\end{itemize}

\noindent
In short, the sign of the vote-share error \(\Delta_i(p)\) depends on (a) how municipality~\(i\)’s mail-in votes for party~\(p\) compare to the district average (driving the numerator difference), and (b) the relative changes in mail-in votes for all parties in the denominator. Municipalities (and parties) whose true mail-in ratios exceed the district-wide average are generally \emph{under-allocated}, producing \(\Delta_i(p)<0\). Conversely, those below the district average end up \emph{over-allocated} with \(\Delta_i(p)>0\). Because the \emph{sums of vote-level errors} across all parties in a district is zero, there is no net \emph{district-level} over- or undercount. But for any single party \emph{in a particular municipality}, the vote-share error can be systematically positive or negative, depending on the local/district mismatch in mail-in voting patterns.

\end{document}