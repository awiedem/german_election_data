\documentclass[12pt]{article}

% Essential packages for math and symbols
\usepackage{amsmath}
\usepackage{amssymb}
\usepackage{amsfonts}
\usepackage{mathtools}

% For better typography and language support
\usepackage[utf8]{inputenc}
\usepackage[T1]{fontenc}
\usepackage{microtype}

% For enumerated lists and itemize environments
\usepackage{enumitem}

% For custom math operators and symbols
\DeclareMathOperator*{\argmin}{arg\,min}
\DeclareMathOperator*{\argmax}{arg\,max}

% Custom math commands for frequently used notation
\newcommand{\R}{\mathbb{R}}
\newcommand{\N}{\mathbb{N}}
\newcommand{\ip}{\text{ip}}
\newcommand{\mail}{\text{mail}}
\newcommand{\total}{\text{total}}

% Margins 1 inch
\usepackage[letterpaper, margin=1in]{geometry}

% For figures
\usepackage{graphicx}

\begin{document}

\section{General Problem Overview}

In our setting, some municipalities share mail-in districts so that their mail-in votes are only observed in aggregated form at the district level. Let $\mathcal{S}$ be the set of municipalities that \emph{share} a mail-in district with at least one other municipality, and let $\mathcal{I}$ be the set of municipalities that do \emph{not} share a mail-in district (i.e., each municipality in $\mathcal{I}$ has a unique mail-in vote count). For each municipality $i$, let $w_i$ be the number of polling card voters, $Y_i^{\ip}$ its in-person votes, and $Y_i^{\mail}$ its observed mail-in votes if it does not share a district (i.e., $i \in \mathcal{I}$). 

When multiple municipalities share a district, we only observe their collective mail-in votes,
\[
    Y_d^{\mail} \;=\; \sum_{i \in d} Y_i^{\mail},
\]
where $d$ indexes the shared mail-in district. To recover municipality-level mail-in votes for each $i \in d$, we implement a \emph{distribution procedure} that allocates $Y_d^{\mail}$ proportionally to each municipality's share of polling card voters within the district:
\[
    \widetilde{Y}_i^{\mail}
    \;=\;
    \frac{w_i}{\sum_{j \in d} w_j}
    \; Y_d^{\mail}.
\]
This ensures that every municipality receives a fraction of the total mail-in votes proportional to its fraction of polling card voters. However, the actual mail-in votes for each municipality may not precisely match this proportional split, potentially introducing a \emph{measurement error}. 

In what follows, we develop an approach to quantify this measurement error and assess how robust our polling-card-based allocation is under different district configurations.

\section{Measurement Error Setup}

To evaluate the measurement error introduced by our mail-in vote distribution procedure, we construct \emph{artificial} mail-in districts and compare municipalities’ \emph{total} votes (in-person plus mail-in) under our distribution approach versus their \emph{observed} total votes. Let $\mathcal{I}$ be the set of municipalities that \emph{do not} share a mail-in district (i.e., each municipality $i \in \mathcal{I}$ has its own unique mail-in vote count). Denote by $w_i$ the number of polling card voters in municipality~$i$, let $Y_i^{\ip}$ be its in-person votes, and $Y_i^{\mail}$ its observed mail-in votes. Hence, the \emph{observed total} votes are 
\[
    Y_i^{\total} \;=\; Y_i^{\ip} \;+\; Y_i^{\mail}.
\]
We proceed as follows:

\begin{enumerate}
    \item \textbf{Subset the data:} Restrict the dataset to all municipalities $i \in \mathcal{I}$. For each of these, we observe $Y_i^{\ip}$, $Y_i^{\mail}$, and $w_i$.

    \item \textbf{Create $D$ artificial mail-in districts:} Partition the municipalities in $\mathcal{I}$ into $D$ groups (districts). We consider two approaches:
    \begin{itemize}
        \item \textit{Random assignment:} Assign each municipality $i$ to one of the $D$ districts with equal probability.
        \item \textit{K-means clustering:} Use centroid coordinates of municipalities to create spatially contiguous districts via a $k$-means algorithm. This more closely reflects the geographic adjacency typical of real mail-in districts.
    \end{itemize}

    \item \textbf{Aggregate mail-in votes and redistribute:} For each artificially created district $d \in \{1,\dots,D\}$, let
    \[
        Y_d^{\mail} \;=\; \sum_{i \in d} Y_i^{\mail}.
    \]
    We then redistribute $Y_d^{\mail}$ back to each municipality $i \in d$ proportionally to its $w_i$:
    \[
        \widetilde{Y}_i^{\mail} 
        \;=\; 
        \frac{w_i}{\sum_{j \in d} w_j} 
        \; Y_d^{\mail},
    \]
    mirroring our procedure in Section \ref{sec:mailin}. Next, we form an \emph{artificial} total vote count for $i$:
    \[
        \widetilde{Y}_i^{\total} 
        \;=\; 
        Y_i^{\ip} 
        \;+\;
        \widetilde{Y}_i^{\mail}.
    \]

    \item \textbf{Compare observed vs.\ artificial totals:} For each municipality $i$, compute the discrepancy
    \[
        \Delta_i 
        \;=\; 
        \widetilde{Y}_i^{\total}
        \;-\; 
        Y_i^{\total}
        \;=\;
        \widetilde{Y}_i^{\mail} \;-\; Y_i^{\mail}.
    \]
    This measures how much the \emph{artificial} total deviates from the \emph{observed} total. It offers a direct estimate of the measurement error induced by our polling-card-based allocation method under different districting scenarios.
\end{enumerate}

\section{Measurement Error by Party and Method}

We consider $D=1{,}000$ artificially created mail-in districts. For this number of districts, we approximate the average number of munis in the ``real'' districts. However, average district population is higher since the set of munis that do not share a mail-in district has a higher mean population than the set of munis that do share a mail-in district.

\emph{Weighted} measurement error means that we weight $\Delta_i$ by $Y_i^{\ip} + \widetilde{Y}_i^{\mail}$ (the municipal-level total based on distributed mail-in votes), while \emph{unweighted} uses no weighting. Note that $Y_i^{\ip} + \widetilde{Y}_i^{\mail}$ is not the same as $Y_i^{\ip} + Y_i^{\mail}$ (i.e., we use \emph{distributed} rather than \emph{observed} mail-in votes).

Figure~\ref{fig:measurement_error_allocation_weighted} shows $\Delta_i$ for both district generation methods (random and $k$-means) across parties. The error bars denote $\pm 1$ standard deviation. Since we compute $\Delta_i$ separately by party, the figures depict party-specific discrepancies. As noted by Florian, the weighted mean of $\Delta_i$ is very close to $0$; small deviations from $0$ likely come from rounding. 

\begin{figure}[!h]
\centering
\includegraphics[width=0.9\textwidth]{../output/figures/measurement_error_allocation_weighted.pdf}
\caption{Measurement error $\Delta_i$ by party and method with $D=1{,}000$ districts. ``Weighted'' errors weight the mean of $\Delta_i$ by $Y_i^{\ip} + \widetilde{Y}_i^{\mail}$. Error bars are $\pm 1$\,SD.}
\label{fig:measurement_error_allocation_weighted}
\end{figure}

\section{Measurement Error by Population, Method, and Party}

We also examine $\Delta_i$ for varying polling card voter thresholds, for 1,000 districts. Figure~\ref{fig:measurement_error_allocation_a2_a3} shows unweighted errors among municipalities below the polling card voter threshold on the $x$-axis, for both random and $k$-means districting. Error bars represent $\pm 1$\,SD.

\begin{figure}[!h]
\centering
\includegraphics[width=0.8\textwidth]{../output/figures/measurement_error_allocation_a2_a3.pdf}
\caption{Unweighted $\Delta_i$ vs.\ polling card voter threshold. Random vs.\ $k$-means districting for municipalities below each threshold on the $x$-axis. Error bars are $\pm 1$\,SD.}
\label{fig:measurement_error_allocation_a2_a3}
\end{figure}

Figure~\ref{fig:measurement_error_allocation_a2_a3_weighted} then shows the same analysis but with \emph{weighted} $\Delta_i$, where the weight is $Y_i^{\ip} + \widetilde{Y}_i^{\mail}$. This again differs from $Y_i^{\ip} + Y_i^{\mail}$. We compare both random and $k$-means district assignments; error bars denote $\pm 1$\,SD. The threshold on the $x$-axis is again based on $w_i$, the number of polling card voters in each municipality.

Finally, note that the weighted $\Delta_i$ mean does not necessarily have to be close to $0$ since the results below are for \textit{subsets} of municipalities, which means that deviations in either direction do not cancel out.

\begin{figure}[!h]
\centering
\includegraphics[width=0.8\textwidth]{../output/figures/measurement_error_allocation_a2_a3_weighted.pdf}
\caption{Weighted $\Delta_i$ vs.\ polling card voter threshold, by random vs.\ $k$-means districting for municipality-party combinations below each threshold on the $x$-axis. Error bars are $\pm 1$\,SD.}
\label{fig:measurement_error_allocation_a2_a3_weighted}
\end{figure}

\section{Approaches to Reduce Measurement Error}

We conclude by outlining potential strategies to reduce the measurement error introduced by our mail-in vote redistribution procedure. Let $\widetilde{Y}_i^{\mail}$ denote the currently assigned mail-in votes for municipality $i$ under the proportional allocation, and let $Y_i^{\mail}$ be the true (but unobserved, if $i$ shares a district) mail-in votes.

\begin{enumerate}[label=(\alph*)]
    \item \textbf{Incorporate Additional Covariates into the Allocation Rule.} 
    Instead of relying solely on $w_i$ (the polling card voter count) to distribute $Y_d^{\mail}$, one could use further covariates such as past election outcomes, demographic structure, or party-specific turnout rates. Formally, for a shared mail-in district $d$, one might replace 
    \[
        \frac{w_i}{\sum_{j \in d} w_j}
    \]
    with a function 
    \[
        f\bigl(w_i, x_i\bigr) \;=\; \frac{g\bigl(w_i, x_i\bigr)}{\sum_{j \in d} g\bigl(w_j, x_j\bigr)},
    \]
    where $x_i$ is a vector of additional predictors (e.g., demographic indicators) and $g(\cdot)$ is chosen to reflect how mail-in votes scale with $w_i$ and $x_i$. This refined allocation rule may better capture variation across municipalities and reduce systematic allocation errors.

    \item \textbf{Leverage Small-Area Estimation or Hierarchical Modeling.}
    One can embed the mail-in vote allocation into a small-area estimation framework that pools information across comparable municipalities. For instance, consider a hierarchical model:
    \[
        Y_i^{\mail} \;=\; \theta_i \cdot Y_d^{\mail}
        \quad\text{for } i \in d,
    \]
    where $\theta_i$ is modeled using partial pooling based on known covariates, prior distributions, or historical data. By allowing $\theta_i$ to vary according to municipality-level or region-level characteristics, the estimates of $\widetilde{Y}_i^{\mail}$ (the predicted mail-in votes) can better reflect actual voting patterns within each district.

    \item \textbf{Comparing Methods and Recommendations.}
    Although many approaches can be explored, we highlight two that are likely to yield substantive improvements:
    \begin{enumerate}[label=(\roman*)]
        \item \emph{Refined Covariate-Based Allocation (Method a)}: This method is straightforward to implement, can incorporate different sets of predictors, and still maintains a proportional-type rule that scales mail-in votes by municipality size and other relevant variables.
        \item \emph{Hierarchical Small-Area Estimation (Method b)}: This approach can capture additional structure—particularly when there is information about comparable municipalities or historical elections. It also provides a principled way to borrow strength across similar areas and shrink extreme estimates toward more reasonable values.
    \end{enumerate}
    Both methods can be deployed with modest changes to the existing allocation framework, and either can reduce measurement error by accounting for the fact that $w_i$ alone may not fully reflect $Y_i^{\mail}$ in shared mail-in districts.
\end{enumerate}

\end{document}